\section{Background}
\subsection{Microservice Testing Challenges}\label{sec:bg-challenges}
Industry surveys and secondary studies consistently highlight that
\emph{testing is among the most costly and error–prone activities once
a system adopts microservice architecture}~\cite{waseem2020testing, miao2025sms}.  
Three dimensions are recurrent across the literature:

\begin{enumerate}[leftmargin=*,label=\roman*)]
  \item \textbf{API correctness} — Each service exposes a RESTful (or
        gRPC/GraphQL) contract whose input domain must be exercised
        thoroughly; malformed requests or forgotten edge cases may
        propagate faults to upstream clients.
  \item \textbf{Workflow integrity} — Real business transactions
        traverse \emph{multiple} services.  
        Integration‐relevant faults such as data-flow mismatches,
        missing authentication tokens, or out-of-order requests represent
        a significant proportion of production incidents in recent
        studies~\cite{zhou2018microservice}.
  \item \textbf{Resilience under partial failure} — Network partitions,
        container restarts, and eventual-consistency delays make fault
        injection (a.k.a.\ chaos testing) indispensable.  Mapping
        studies report that fewer than one-third of automated tools
        provide built-in support for resilience
        validation~\cite{li2020testing}.
\end{enumerate}

The \emph{Systematic Mapping Study of Test Generation for
Microservices}~\cite{miao2025sms} further classifies existing work
into \textit{consumer-driven contract}, \textit{search-based input
generation}, and \textit{resilience testing}; yet it concludes that
no current tool unifies these approaches into a single, end-to-end
workflow-aware framework.

\subsection{The \textsc{RESTest} Framework}\label{sec:restest}
\textsc{RESTest} is an open-source, \emph{model-based}, black-box
testing framework that derives executable test cases directly from an
OpenAPI specification~\cite{martin2019restest}.  
Its core pipeline (Figure 1 in~\cite{martin2019restest}) comprises:

\begin{enumerate}[leftmargin=*,label=\arabic*)]
  \item \textbf{Test-model generation} — Converts the OpenAPI document
        into an internal YAML test model enriched with
        constraints (e.g.\ regular expressions, numeric ranges).
  \item \textbf{Test-case generation} — Applies a portfolio of
        techniques such as fuzzing, adaptive random testing, and
        constraint solving to maximise branch and HTTP-state coverage.
  \item \textbf{Test writing} — Instantiates the abstract tests into
        concrete clients (e.g.\ \textit{REST-assured}, Postman).
  \item \textbf{Execution \& reporting} — Runs the tests online or
        offline and produces Allure reports.
\end{enumerate}

\textbf{Extensibility.}  
\textsc{RESTest} exposes \emph{plug-in interfaces} for custom data
generators and writers, enabling researchers to prototype new
strategies without modifying the core engine.  
Prior extensions include grammar-based generators, OAuth2
authenticators, and domain-specific value providers, but \emph{none}
addresses cross-endpoint data flow or multi-service orchestration
out-of-the-box.

\textbf{Positioning of our work.}  
The generators referenced throughout this paper—
\textit{AiDrivenLLMGenerator}, \textit{SmartSemanticTestCaseGenerator},
and \textit{SchemaGuidedDependencyMutator}—\emph{are new plug-ins we
developed} to extend \textsc{RESTest}.  
They complement, rather than replace, the original
constraint-based and fuzzing engines reported by
Martin-López \emph{et al.}~\cite{martin2019restest}.  
Our contribution is therefore orthogonal: we retain the proven
single-endpoint capabilities of \textsc{RESTest} while adding a
trace-driven layer for \emph{workflow-level} generation and execution.


%====================================================================
%====================================================================
\section{Methodology}\label{sec:method}
This section presents, at an architectural and algorithmic level, how the proposed multi-service extension for \textbf{RESTest} transforms distributed traces into dependency-aware, executable \emph{workflow test cases}.  The design preserves RESTest’s proven \textit{generate → write → execute → report} pipeline while adding trace-driven scenario extraction and cross-service dependency management.

%====================================================================
\subsection{Architecture Overview}\label{ssec:arch}
The extended pipeline introduces four dedicated subsystems, orchestrated by
the core RESTest workflow engine:

\begin{enumerate}[leftmargin=*]
  \item \textbf{Trace Workflow Extractor}  
        \hfill(\S\ref{ssec:extract})\\
        Parses OpenTelemetry/Jaeger span logs and reconstructs hierarchical
        \textsc{WorkflowScenario}s (Algorithm~\ref{alg:extract}).
  \item \textbf{Multi-Service Test Generator}  
        \hfill(\S\ref{ssec:gen})\\
        Converts scenarios into concrete \textsc{MultiServiceTestCase}s, combining
        LLM-driven parameter synthesis with trace-based dependency resolution
        (Algorithm~\ref{alg:generate}).
  \item \textbf{Multi-Service Test Writer}\\
        Emits dependency-chained JUnit/REST-assured classes ready for
        compilation and execution.
  \item \textbf{RESTest Execution Engine}\\
        Re-uses RESTest’s compilation, JUnit execution and Allure/CSV reporting
        stack, extended with step-level dependency orchestration.
\end{enumerate}

\begin{figure}[t]
  \centering
  \includegraphics[width=\linewidth]{figs/mst-architecture.png}
  \caption{End-to-end data-flow across the extended RESTest pipeline for
           multi-service testing.  Inputs (\textit{blue}) feed the processing
           stage (\textit{orange}), which in turn produces executable artefacts
           consumed by the execution and reporting stage (\textit{green}).  The
           black arrows denote feedback loops used for dependency injection
           and result aggregation.}
  \label{fig:dataflow}
\end{figure}

\noindent%
Figure~\ref{fig:dataflow} replaces the earlier text-based graph and should be
exported (e.g., as \texttt{mst-architecture.png} or PDF) at publication
quality.  The descriptive caption ensures accessibility for print media and
screen readers.

\subsubsection*{Scalability and Fault Isolation}
\begin{itemize}[leftmargin=*]
  \item \emph{Parallelism:} trace parsing, scenario generation, and JUnit runs
        execute in parallel—bounded by CPU cores—via RESTest’s existing
        executor pool.
  \item \emph{Graceful Degradation:} failures in one stage propagate as
        structured Allure events without aborting the entire pipeline, enabling
        incremental debugging of complex workflows.
\end{itemize}

%--------------------------------------------------------------------
\subsection{Trace-Driven Workflow Extraction}
\label{ssec:extract}
\begin{algorithm}[tb]
\caption{TRACE\_WORKFLOW\_EXTRACTOR}\label{alg:extract}
\footnotesize
\begin{algorithmic}[1]
\Require\;{Path(s) to span log(s)}
\Ensure\;{List of \textsc{WorkflowScenario}s}
\State Parse span JSON/JSONL into objects
\State Group spans by \texttt{traceId} $\rightarrow$ map $G$
\ForAll{trace $t \in G$}
    \State Sort $G[t]$ by \texttt{startTime}
    \State Initialise empty scenario $w$
    \ForAll{span $s \in G[t]$}
        \State $\sigma \leftarrow$ \textproc{SpanToStep}$(s)$
        \State Insert $\sigma$ into $w$ respecting parent–child links
    \EndFor
    \State $w.dep\_map \leftarrow$ \textproc{InferDependencies}$(w)$
    \State Append $w$ to scenario list
\EndFor
\State Scenario list $\gets$ \textproc{MergeCrossTraceDeps}(list)
\State \Return scenario list
\end{algorithmic}
\end{algorithm}

\paragraph{Span Processing.}
Each span is mapped to a \textsc{WorkflowStep} containing service metadata,
HTTP method/path, request/response bodies, status, and timing.  Parent–child
links reconstruct the call hierarchy within a trace.

\paragraph{Cross-Trace Merging.}
\textproc{MergeCrossTraceDeps} joins scenarios whose output values (e.g.,
\texttt{orderId}) appear as inputs in subsequent traces, producing unified
end-to-end workflows.

%--------------------------------------------------------------------
\subsection{Multi-Service Test Generation}
\label{ssec:gen}

The generator transforms abstract \textsc{WorkflowScenario}s into concrete
\textsc{MultiServiceTestCase}s through a three-phase process: parameter
synthesis, dependency resolution, and variant generation.  This approach
balances trace fidelity with test diversity while maintaining workflow
consistency.

\begin{algorithm}[tb]
\caption{MULTI\_SERVICE\_TEST\_GENERATOR}\label{alg:generate}
\footnotesize
\begin{algorithmic}[1]
\Require\;{Scenario $w$, budget $B$, service configs $\mathcal{C}$}
\Ensure\;{Set of \textsc{MultiServiceTestCase}s}
\State $T \gets \{$baseline copy of $w$$\}$
\For{$i \gets 1$ to $|w.steps|$}
    \For{each parameter $p$ in $w.steps_i$}
        \State $S_{\textsc{llm}} \gets$ \textproc{LLMSynthesis}$(p, \mathcal{C}, 5)$ \Comment{Generate 5 seed values}
        \State $V_{\textsc{w2v}} \gets$ \textproc{Word2VecExpansion}$(S_{\textsc{llm}})$
        \State $V_{\textsc{bert}} \gets$ \textproc{BERTExpansion}$(S_{\textsc{llm}})$
        \State $V_{\textsc{mut}} \gets$ \textproc{SchemaGuidedMutation}$(S_{\textsc{llm}})$
        \State $V_p \gets S_{\textsc{llm}} \cup V_{\textsc{w2v}} \cup V_{\textsc{bert}} \cup V_{\textsc{mut}}$
    \EndFor
    \State $V \gets$ \textproc{SelectVariants}$(\text{Cartesian product of all } V_p, B-|T|)$
    \ForAll{variant combination $v \in V$}
        \State $w' \gets$ deep copy of $w$; substitute $v$ in step $i$
        \State \textproc{PropagateDeps}$(w', i, \mathcal{C})$
        \State \textproc{ValidateWorkflow}$(w')$
        \State $T \gets T \cup \{w'\}$
    \EndFor
\EndFor
\State \Return $T$
\end{algorithmic}
\end{algorithm}

\paragraph{Parameter Synthesis Strategy.}
The generator employs a two-stage approach that maximizes test diversity
while minimizing computational overhead from expensive LLM calls:

\textbf{Stage 1: LLM-Driven Initial Synthesis} uses the
\emph{AiDrivenLLMGenerator} to generate an initial set of realistic
parameter values (typically 5) based on OpenAPI schema metadata.  The LLM
receives a structured prompt containing:
\begin{itemize}[leftmargin=*]
  \item Parameter metadata (name, type, format, constraints, examples)
  \item OpenAPI schema information (pattern, enum values, descriptions)
  \item Context information (HTTP method, service name, operation)
  \item Generation instructions (number of values, format requirements)
\end{itemize}

\textbf{Stage 2: Semantic Expansion} takes the LLM-generated seed values
and expands them using computationally efficient semantic models to
generate additional test variants:

\begin{enumerate}[leftmargin=*]
  \item \textbf{Word2Vec Expansion:} For each LLM-generated value,
        queries the Word2Vec model to find semantically similar terms,
        creating variants that maintain contextual relevance
  \item \textbf{BERT-Based Token Replacement:} Applies masked language
        modeling to substitute tokens within the LLM-generated values,
        producing context-aware variations
  \item \textbf{Schema-Guided Mutation:} Uses the
        \emph{SchemaGuidedDependencyMutator} to apply systematic
        transformations including boundary values, type-specific
        mutations, and constraint violations
\end{enumerate}

This two-stage approach addresses the resource limitation of running LLMs
locally by generating high-quality initial values once, then leveraging
cheaper semantic models to achieve the desired test case volume.

\paragraph{Dependency Resolution and Propagation.}
The \textproc{PropagateDeps} procedure ensures workflow consistency by
tracking data dependencies across service boundaries:

\begin{enumerate}[leftmargin=*]
  \item \textbf{Dependency Graph Construction:} Analyzes each step's
        input/output fields to build a directed acyclic graph (DAG)
        representing data flow dependencies
  \item \textbf{Context Map Maintenance:} Maintains a runtime context
        mapping output field names to their current values, updated
        after each step execution
  \item \textbf{Parameter Injection:} For each step $i$, scans its
        parameters for placeholders (e.g., \texttt{\$\{orderId\}}) and
        substitutes them with values from the context map
  \item \textbf{Cross-Service Validation:} Verifies that injected
        values conform to the target service's schema constraints
\end{enumerate}

\paragraph{Variant Selection and Budget Management.}
The \textproc{SelectVariants} function implements a multi-objective
optimization strategy that balances:

\begin{itemize}[leftmargin=*]
  \item \textbf{Diversity:} Maximizes parameter space coverage using
        distance metrics between variant sets
  \item \textbf{Validity:} Prioritizes variants that pass schema
        validation and business logic checks
  \item \textbf{Relevance:} Favors variants that maintain semantic
        coherence with the original workflow
  \item \textbf{Complexity:} Distributes test complexity across
        different failure modes and edge cases
\end{itemize}

The selection algorithm uses a weighted scoring function that considers
variant quality, execution cost, and coverage contribution to the overall
test suite.

%--------------------------------------------------------------------
\subsection{Test Execution and Validation}
\label{ssec:exec}

The execution engine orchestrates multi-service test cases through a
sophisticated pipeline that handles dependency resolution, fault isolation,
and comprehensive result validation.  Unlike traditional single-service
testing, the executor must maintain workflow state across multiple service
boundaries while providing detailed failure diagnostics.

\begin{algorithm}[tb]
\caption{MULTI\_SERVICE\_TEST\_EXECUTION}\label{alg:execute}
\footnotesize
\begin{algorithmic}[1]
\Require\;{Set of test cases $\mathcal{T}$, service configs $\mathcal{C}$}
\Ensure\;{Execution results with failure diagnostics}
\ForAll{$tc \in \mathcal{T}$}
    \State Compile $tc$ to Java class; register with JUnit framework
    \State Initialize execution context: $\text{ctx} \gets \{\}$
    \ForAll{step $s \in tc.steps$}
        \State $\text{params} \gets$ \textproc{ResolveParameters}$(s, \text{ctx}, \mathcal{C})$
        \State $\text{headers} \gets$ \textproc{BuildHeaders}$(s, \text{ctx})$
        \State $(code, body, \text{traceId}) \gets$ \textproc{HTTP\_CALL}$(s, \text{params}, \text{headers})$
        \If{$\textproc{ValidateResponse}(code, body, s.expected)$}
            \State $\text{ctx} \gets$ \textproc{ExtractOutputs}$(body, s.output\_schema)$
            \State \textproc{UpdateDependencyMap}$(\text{ctx}, s)$
        \Else
            \State \textproc{HandleFailure}$(s, code, body, \text{ctx})$
            \State Mark dependent steps as skipped
        \EndIf
        \If{$\text{traceId} \neq \text{null}$}
            \State \textproc{ValidateTrace}$(\text{traceId}, s.expected\_trace)$
        \EndIf
    \EndFor
    \State \textproc{EmitResults}$(tc, \text{ctx})$
\EndFor
\end{algorithmic}
\end{algorithm}

\paragraph{Parameter Resolution and Context Management.}
The \textproc{ResolveParameters} function implements sophisticated
dependency resolution that handles multiple data flow patterns:

\textbf{Direct Field Mapping:} Maps output fields from previous steps
to input parameters using exact name matching (e.g., \texttt{orderId}
from step 1 to \texttt{orderId} in step 2).

\textbf{Schema-Based Extraction:} For complex nested responses, uses
JSONPath expressions to extract specific fields (e.g.,
\texttt{\$.data.items[0].id}).

\textbf{Transformation Functions:} Applies type conversions and
formatting operations (e.g., converting timestamps, encoding URLs,
formatting currency values).

\textbf{Default Value Injection:} Provides fallback values for
optional parameters that weren't produced by previous steps.

The context management system maintains a hierarchical namespace that
prevents naming conflicts between different services while enabling
cross-service data sharing.

\paragraph{Response Validation and Failure Handling.}
The \textproc{ValidateResponse} function implements multi-level
validation that goes beyond simple status code checking:

\begin{enumerate}[leftmargin=*]
  \item \textbf{HTTP-Level Validation:} Verifies status codes,
        headers, and response timing against expected values
  \item \textbf{Schema Validation:} Validates response body against
        OpenAPI schema definitions using JSON Schema validation
  \item \textbf{Business Logic Validation:} Applies domain-specific
        rules (e.g., order totals must be positive, dates must be
        in chronological order)
  \item \textbf{Cross-Service Consistency:} Verifies that data
        produced by one service is consistent with expectations
        of downstream services
\end{enumerate}

When validation fails, the \textproc{HandleFailure} function provides
detailed diagnostics including:
\begin{itemize}[leftmargin=*]
  \item Exact failure location (step, parameter, field)
  \item Expected vs. actual values with diffs
  \item Dependency chain analysis showing which previous steps
        contributed to the failure
  \item Suggested debugging actions and common failure patterns
\end{itemize}

\paragraph{Trace-Based Oracle Validation.}
When services return trace identifiers, the executor performs
comprehensive trace validation through \textproc{ValidateTrace}:

\begin{itemize}[leftmargin=*]
  \item \textbf{Call Graph Verification:} Ensures all expected
        internal service calls occurred in the correct sequence
  \item \textbf{Performance Analysis:} Validates response times
        and identifies performance bottlenecks
  \item \textbf{Error Propagation Tracking:} Maps error conditions
        across service boundaries to identify root causes
  \item \textbf{Data Flow Validation:} Verifies that data passed
        between internal services matches expected patterns
\end{itemize}

This trace-based validation provides significantly stronger oracle
capabilities than traditional HTTP-level testing, enabling detection
of integration issues that would otherwise remain hidden.

\paragraph{Result Aggregation and Reporting.}
The \textproc{EmitResults} function produces comprehensive execution
reports that integrate with RESTest's existing reporting infrastructure:

\begin{itemize}[leftmargin=*]
  \item \textbf{Allure Integration:} Generates structured XML reports
        with step-by-step execution details, screenshots, and
        failure diagnostics
  \item \textbf{CSV Statistics:} Produces machine-readable summaries
        for automated analysis and trend tracking
  \item \textbf{Workflow Visualization:} Creates dependency graphs
        showing successful and failed execution paths
  \item \textbf{Performance Metrics:} Tracks execution times,
        throughput, and resource utilization across services
\end{itemize}

The reporting system maintains backward compatibility with existing
RESTest dashboards while adding multi-service specific visualizations
and metrics.

%--------------------------------------------------------------------
\subsection{Integration Points with RESTest}
\begin{itemize}[leftmargin=*]
  \item \textbf{Configuration.}  YAML schema extended with multi-service
        mappings; legacy single-service options remain untouched.
  \item \textbf{Execution.}  Plug-in writer registers generated classes with
        RESTest’s \texttt{WorkflowExecutor}; no changes to the core runner.
  \item \textbf{Reporting.}  Existing Allure and statistics managers remain
        unchanged, ensuring homogeneous result dashboards.
\end{itemize}

\subsection{Coverage Analysis Limitations}\label{sec:coverage-limitations}

Traditional API testing coverage models, as implemented in RESTest's \textsc{CoverageMeter}, are designed for single-service scenarios and face fundamental challenges when applied to multi-service workflow testing. This section details why coverage analysis is currently disabled for MST mode and the research challenges involved.

\paragraph{Single-Service Coverage Model Assumptions.}
RESTest's coverage framework follows established API testing coverage criteria \cite{martin2019restest}, including:
\begin{itemize}[leftmargin=*]
    \item \textit{Parameter Coverage:} Ensuring all input parameters are exercised
    \item \textit{Path Coverage:} Testing all API endpoints and URL templates  
    \item \textit{Operation Coverage:} Exercising all HTTP methods (GET, POST, etc.)
    \item \textit{Status Code Coverage:} Triggering different response codes
    \item \textit{Response Body Coverage:} Validating output schema elements
\end{itemize}

These criteria assume a unified OpenAPI specification and coherent parameter space, which breaks down in multi-service environments where each service has independent schemas and operation sets.

\paragraph{Multi-Service Coverage Challenges.}
Extending coverage models to distributed workflows introduces several theoretical and practical challenges:

\textbf{1. Cross-Service Dependency Coverage:} Traditional coverage metrics do not account for data flow between services. A parameter may be "covered" in one service but its downstream usage in dependent services remains unmeasured \cite{zhou2018microservice}.

\textbf{2. Workflow Path Coverage:} While single-service path coverage considers URL endpoints, multi-service scenarios require modeling paths through \emph{service interaction graphs}, where the same logical workflow can traverse different service combinations based on system state \cite{li2020testing}.

\textbf{3. Temporal Coverage Semantics:} Multi-service workflows involve asynchronous interactions and eventual consistency, making it unclear when coverage criteria are satisfied. A service call may succeed initially but fail in downstream processing, complicating coverage attribution \cite{fowler2014microservices}.

\textbf{4. Configuration Space Explosion:} With $n$ services each having independent configuration parameters, the joint parameter space grows exponentially, making comprehensive coverage computationally intractable \cite{arcuri2019restful}.

\paragraph{Implementation Challenges.}
The current RESTest architecture expects a single \textsc{CoverageGatherer} instance built from one OpenAPI specification. Multi-service testing requires:
\begin{itemize}[leftmargin=*]
    \item Multiple service specifications with potentially conflicting schema definitions
    \item Cross-service parameter dependency tracking
    \item Workflow-level coverage criteria that span service boundaries
    \item New coverage aggregation strategies for distributed test results
\end{itemize}

\paragraph{Future Research Directions.}
Addressing multi-service coverage requires developing new theoretical frameworks that:
\begin{enumerate}[leftmargin=*]
    \item Define workflow-level coverage criteria that consider end-to-end user scenarios rather than individual API operations \cite{daniel2023testing}
    \item Model coverage as graphs where nodes represent service operations and edges represent data dependencies
    \item Incorporate temporal aspects of distributed system behavior into coverage semantics
    \item Develop practical algorithms for coverage measurement that scale to realistic microservice deployments
\end{enumerate}

The current implementation prioritizes functional workflow testing over coverage measurement, acknowledging that meaningful multi-service coverage models remain an active area of research in the software testing community.
